\documentclass[a4paper]{article}
\usepackage[utf8]{inputenc}
\usepackage[T1]{fontenc}
\usepackage{amsmath}
\usepackage{amssymb}
\usepackage{minted}
\usepackage[align]{flatbox}

\usemintedstyle{colorful}
\newminted{latex}{linenos}

\newcommand{\dashskip}{%
\begin{center}
  \rule{3em}{.1pt}
\end{center}}

\BeforeBeginEnvironment{latexcode}{\medskip\textit{\underline{\LaTeX{} code:}}}

\title{Demo of \texttt{flatbox}}
\author{Léo Valais}
\date\today

\begin{document}
\maketitle

Dependencies:
\begin{itemize}
\item xcolor
\item tcolorbox
\item ifthen
\item applecolors (optional)
\end{itemize}

\section{Presentation}
The package \texttt{flabox} provides three kind of \texttt{tcolorbox}es:

\begin{itemize}
\item \texttt{flatbox}: boxes without background
\item \texttt{ffbox}: boxes with filled background
\item \texttt{groupbox}: another kind of minimalistic box
\end{itemize}

\begin{flatbox}{Title}
I'm a \texttt{flatbox}.
\end{flatbox}

\begin{ffbox}{Title}
I'm a \texttt{ffbox}.
\end{ffbox}

\begin{groupbox}{Title}
I'm a \texttt{groupbox}.
\end{groupbox}

\begin{latexcode}
\begin{flatbox}{Title}
I'm a \texttt{flatbox}.
\end{flatbox}

\begin{ffbox}{Title}
I'm a \texttt{ffbox}.
\end{ffbox}

\begin{groupbox}{Title}
I'm a \texttt{groupbox}.
\end{groupbox}
\end{latexcode}

These boxes can have an empty title:

\begin{flatbox}{}
I'm a \texttt{flatbox} without title.
\end{flatbox}

\begin{ffbox}{}
I'm a \texttt{ffbox} without title.
\end{ffbox}

\begin{groupbox}{}
I'm a \texttt{groupbox}.
\end{groupbox}

\begin{latexcode}
\begin{flatbox}{}
I'm a \texttt{flatbox} without title.
\end{flatbox}

\begin{ffbox}{}
I'm a \texttt{ffbox} without title.
\end{ffbox}

\begin{groupbox}{}
I'm a \texttt{groupbox}.
\end{groupbox}
\end{latexcode}

\subsection{Colors}
Since these boxes are intentionally simplistic, the main point one wants to
customize is the colors. To define \texttt{flatbox}-compatible colors, use \texttt{$\backslash$setflatboxcolor}.

\begin{latexcode}
\setflatboxcolor[fg]{hippie}{red} % same as \setflatboxcolor{hippie}{red}
\setflatboxcolor[bg]{hippie}{green}
\end{latexcode}

For \texttt{flatbox}es, the background color is useless and it is not even
necessary to define it.

Then, to use this color combination, pass the name of the combination as an
optional parameter to the boxes:

\begin{latexcode}
\begin{flatbox}[hippie]{Hippie flatbox}
I'm a hippie box.
\end{flatbox}

\begin{ffbox}[hippie]{Hippie flatbox}
I'm a hippie box.
\end{ffbox}
\end{latexcode}

\setflatboxcolor[fg]{hippie}{red} % same as \setflatboxcolor{hippie}{red}
\setflatboxcolor[bg]{hippie}{green}

\begin{flatbox}[hippie]{Hippie flatbox}
I'm a hippie box.
\end{flatbox}

\begin{ffbox}[hippie]{Hippie flatbox}
I'm a hippie box.
\end{ffbox}

\medskip

Alternatively, you can manually define the colors using the \texttt{xcolor}
package facilities:

\begin{latexcode}
\colorlet{flatboxhippie.fg}{red}
\definecolor{flatboxhippie.bg}{RGB}{0,255,0}
\end{latexcode}

\bigskip

The \texttt{flatbox} package defines some default colors for the environments
\texttt{flatbox}, \texttt{exampleflatbox}, \texttt{alertflatbox},
\texttt{noteflatbox}, \texttt{theoremflatbox},
\texttt{ffbox}, \texttt{exampleffbox}, \texttt{alertffbox},
\texttt{noteffbox} and \texttt{theoremffbox} which are shown in section \ref{sec:show}.

\dashskip
You can also override default colors, which names are:
\begin{itemize}
\renewcommand{\line}[1]{\item \texttt{flatbox#1.fg} and \texttt{flatbox#1.bg}}
\line{}
\line{example}
\line{alert}
\line{theorem}
\line{note}
\end{itemize}

\dashskip
If the package \texttt{applecolors} is imported \textcolor{red}{\textbf{before}}
\texttt{flatbox}, the iOS colors are used as default colors instead.

\subsection{Error boxes}
This package also defines the command \texttt{$\backslash$errbox} and the
environment \texttt{errorbox} that uses the colors \texttt{flatboxerror.fg} and
\texttt{flatboxerror.bg}. They're sometimes useful to the writter for debugging
purposes or to leave temporary notes but clearly not meant to be kept in the final
document.

I'm a sentance\errbox{oh crap!} with some mistackes\errbox{not again!}.

\begin{errorbox}
  Fix spelling (aspell?).
\end{errorbox}

\begin{latexcode}
I'm a sentance\errbox{oh crap!} with some mistackes\errbox{not again!}.

\begin{errorbox}
  Fix spelling (aspell?).
\end{errorbox}
\end{latexcode}

\subsection{Package options}
The package \texttt{flatbox} accepts two options:
\begin{itemize}
\item \texttt{rounded}: gives \texttt{ffbox}es rounded corners
\item \texttt{align}: if you wish to use both \texttt{flatbox}es and
  \texttt{ffbox}es in the same document, you might want to have their left magin
  aligned. This option does that (\texttt{flatbox}es have more indentation otherwise).

  This option is used in this document.
\end{itemize}

\newpage
\section{Demo}
\label{sec:show}
Each box just has one (required) parameter: the title (can be empty).

\begin{latexcode}
\begin{style of flatbox}{optional title}
\end{latexcode}

\newcommand{\demo}[2][I'm a demo box.\\Also I like to span over\\multiple\\lines.]{%
  \subsection{\texttt{#2flatbox} and \texttt{#2ffbox}}
  \begin{minipage}[c]{.5\linewidth}
    \begin{#2flatbox}{Demo of \texttt{#2flatbox}}
      #1
    \end{#2flatbox}
  \end{minipage}
  \begin{minipage}[c]{.5\linewidth}
    \begin{#2ffbox}{Demo of \texttt{#2ffbox}}
      #1
    \end{#2ffbox}
  \end{minipage}
  \begin{minipage}[c]{.5\linewidth}
    \begin{#2flatbox}{} #1 \end{#2flatbox}
  \end{minipage}
  \begin{minipage}[c]{.5\linewidth}
    \begin{#2ffbox}{} #1 \end{#2ffbox}
  \end{minipage}
  \begin{minipage}[c]{.5\linewidth}
    \begin{#2groupbox}{Demo of \texttt{#2groupbox}}
      #1
    \end{#2groupbox}
  \end{minipage}
  \begin{minipage}[c]{.5\linewidth}
    \begin{#2groupbox}{} #1 \end{#2groupbox}
  \end{minipage}
}

\demo{}
\demo{example}
\demo{alert}
\demo[Stirling's approximation\\\[
  \forall n \in \mathbb{N}, n! \sim \sqrt{2 \cdot \pi \cdot n} \cdot \left(\frac{n}{e}\right)^n
\]]{theorem}
\demo{note}

\end{document}
